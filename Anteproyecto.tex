\documentclass[12pt,openany,letterpaper]{mitthesis}

%\pdfminorversion=4

\usepackage{lgrind}
\usepackage[margin=3cm]{geometry}
\usepackage[utf8]{inputenc}
\usepackage{colortbl}
\usepackage{color}
\usepackage[spanish]{babel}
\decimalpoint
\usepackage{amsmath,mathrsfs}
\usepackage{etoolbox}
\usepackage{hyperref}
\usepackage{amsfonts}
\usepackage{amssymb}
\usepackage{dsfont}	
\usepackage{graphicx}
\usepackage{pdflscape}
\usepackage{afterpage}
\usepackage{cite}
\usepackage{footnote}
\usepackage{datatool}
\usepackage{pdfpages}
%\usepackage{subfigure}
\usepackage{listings}
\usepackage{caption}
\usepackage{subcaption}
\usepackage[toc,page]{appendix}
\usepackage{multicol}
\usepackage{acronym}
\usepackage{sectsty}
\sectionfont{\fontsize{14}{12}\selectfont}
\chapterfont{\fontsize{18}{12}\selectfont}
\usepackage{tocloft}
\usepackage{float}
\setlength{\parindent}{0pt}
\usepackage{mathtools}
\usepackage{siunitx}
\usepackage{wasysym} % Symbolo de diámetro
\usepackage{multirow}
\usepackage[table,xcdraw]{xcolor}
\usepackage{url}
\usepackage{enumitem}
\usepackage{setspace}
\makeatletter
\def\@makechapterhead#1{%
	{\parindent \z@  \centering
		\ifnum \c@secnumdepth >\m@ne
		\Large \bfseries \setstretch{0.3} \thechapter\vspace{.5\baselineskip}.\ % <-- Chapter # (without "Chapter")
		\fi
		\interlinepenalty\@M
		#1\par\nobreak\vspace{2\baselineskip}% <------------------ Chapter title
}}

\makeatother
\newcommand{\sortitem}[1]{%
  \DTLnewrow{list}% Create a new entry
  \DTLnewdbentry{list}{description}{#1}% Add entry as description
}
\newenvironment{sortedlist}{%
  \DTLifdbexists{list}{\DTLcleardb{list}}{\DTLnewdb{list}}% Create new/discard old list
}{%
  \DTLsort{description}{list}% Sort list
  \begin{itemize}%
    \DTLforeach*{list}{\theDesc=description}{%
      \item \theDesc}% Print each item
  \end{itemize}%
}

\makeatletter
\def\resetMathstrut@{%
  \setbox\z@\hbox{%
    \mathchardef\@tempa\mathcode`\[\relax
    \def\@tempb##1"##2##3{\the\textfont"##3\char"}%
    \expandafter\@tempb\meaning\@tempa \relax
  }%
  \ht\Mathstrutbox@\ht\z@ \dp\Mathstrutbox@\dp\z@}
\makeatother
\begingroup
  \catcode`(\active \xdef({\left\string(}
  \catcode`)\active \xdef){\right\string)}
\endgroup
\mathcode`(="8000 \mathcode`)="8000
%end

\makeatletter
\setlength{\@fptop}{0pt}
\makeatother

\hypersetup{
    bookmarks=true,         % show bookmarks bar
    unicode=false,          % non-Latin characters in Acrobat’s bookmarks
    pdftoolbar=true,        % show Acrobat’s toolbar?
    pdfmenubar=true,        % show Acrobat’s menu?
    pdffitwindow=true,     % window fit to page when opened
    pdfstartview={FitH},    % fits the width of the page to the window
    pdftitle={DISEÑO E IMPLEMENTACIÓN DE UN ESPECTRÓMETRO LINEAL EN EL NÚMERO DE ONDA PARA TOMOGRAFÍA DE COHERENCIA ÓPTICA},% title
    pdfauthor={Sebastian Ruiz Lopera},     % author
    pdfsubject={Trabajo de Grado},   % subject of the document
    %pdfcreator={Creator},   % creator of the document
    %pdfproducer={Producer}, % producer of the document
    pdfkeywords={Espectrometro lineal en k} {tomografía óptica de coherecia} {diseño opto-mecánico}, % list of keywords
    pdfnewwindow=true,      % links in new window
    colorlinks=true,       % false: boxed links; true: colored links
    linkcolor=blue,       % color of internal links (change box color with linkbordercolor)
    citecolor=blue,        % color of links to bibliography
    filecolor=magenta,      % color of file links
    urlcolor=blue,           % color of external links
    linktocpage=true
}

\pagenumbering{roman} % Núemrar páginas con números romanos
\usepackage[open, openlevel=2, atend]{bookmark}
%%%%%%%%%%%%%%%%% Temas y subtemas tentativos en los capítulos %%%%%%%%%%%%%%%%


%%%%%%%%%%%%%%%%%%%%%%%%%%%%%% Empezar documento %%%%%%%%%%%%%%%%%%%%%%%%%%%%%%	

\begin{document}

\renewcommand{\listfigurename}{\Large Lista de figuras}
\renewcommand{\listtablename}{\Large Lista de tablas}
\renewcommand{\contentsname}{\Large Contenido}
\renewcommand{\tablename}{Tabla} 
\renewcommand{\bibname}{\Large Referencias}


%%%%%%%%%%%%%%%%%%%%%%%%%%%%   Insertar capítulos   %%%%%%%%%%%%%%%%%%%%%%%%%%%

%%%%%%%%%%%%%%%%%%%%%%%%%%%%   Cubiertas
\include{Cubierta/Cubierta}

%%%%%%%%%%%%%%%%%%%%%%%%%%%%   Tabla de contenido
\newpage
%\phantomsection
%\addcontentsline{toc}{section}{\bf Contenido}
\currentpdfbookmark{Contenido}{Contenido}
\tableofcontents
\thispagestyle{empty}
\newpage

%\setlength{\cftfigindent}{0pt}  % remove indentation from figures in lof
%\setlength{\cfttabindent}{0pt}  % remove indentation from tables in lot

\newpage
\phantomsection
%\addcontentsline{toc}{chapter}{\vspace*{-.5\baselineskip}\bf Lista de figuras}
\addcontentsline{toc}{chapter}{\bf Lista de figuras}
\listoffigures
\thispagestyle{empty}
\newpage

\newpage
\phantomsection
%\addcontentsline{toc}{chapter}{\vspace*{-.5\baselineskip}\bf Lista de tablas}
\addcontentsline{toc}{chapter}{\bf Lista de tablas}
\listoftables
\newpage
\thispagestyle{empty}

%\tableofcontents
%%\thispagestyle{empty}
%\newpage
%\addcontentsline{toc}{section}{\bf Lista de figuras}
%\listoffigures
%%\thispagestyle{empty}
%\newpage
%\addcontentsline{toc}{section}{\bf Lista de tablas}
%\listoftables
%%\thispagestyle{empty}
%\newpage

\newpage
\phantomsection
%\addcontentsline{toc}{chapter}{\vspace*{-.5\baselineskip}\bf Lista de acrónimos}
\addcontentsline{toc}{chapter}{\bf Lista de acrónimos}
\chapter*{Lista de acrónimos}
\begin{acronym}[AWGN]
	
\acro{APDL}{\textit{ANSYS parametric design language} (Lenguaje paramétrico de diseño de ANSYS).}
\acro{BK}{\textit{Borosilicate crown} (Borosilicato crown).}
\acro{F}{\textit{Flint} (Flint).}
\acro{Fig}{\textit{Figure} (Figura).}
\acro{Eq}{\textit{Equation} (Ecuación).}
\acro{FC}{\textit{Fiber coupler} (Acoplador de fibra óptica).}
\acro{FD--OCT}{\textit{Fourier-domain optical coherence tomography} (Tomografía de coherencia óptica en el dominio de Fourier).}
\acro{FEM}{\textit{Finite element method} (Metódo de elementos finitos).}
\acro{FSR}{\textit{Free spectral range} (Rango espectral libre).}
\acro{FWHM}{\textit{Full width at half maximum} (Anchura a media altura).}
\acro{MTF}{\textit{Modulation transfer function} (Función de transferencia de modulación).}
\acro{NIR}{\textit{Near infrared} (Infrarrojo cercano).}
\acro{OCM}{\textit{Optical coherence microscopy} (Microscopía de coherencia óptica).}
\acro{OCT}{\textit{Optical coherence tomography} (Tomografía de coherencia óptica).}
\acro{OFDI}{\textit{Optical frequency domain interferometry} (Interferometría óptica en el dominio frecuencial).}
\acro{OSA}{\textit{Optical spectrum analyzer} (Analizador de espectro óptico).}
\acro{OTF}{\textit{Optical transfer function} (Función de transferencia óptica).}
\acro{PSF}{\textit{Point spread function} (Función de dispersión de punto).}
\acro{RMS}{\textit{Root mean square} (Media cuadrática).}
\acro{SD--OCT}{\textit{Spectral-domain optical coherence tomography} (Tomografía de coherencia óptica en el dominio espectral).}
\acro{SF}{\textit{Dense flint} (Flint denso).}
\acro{SS--OCT}{\textit{Swept-source optical coherence tomography} (Tomografía de coherencia óptica de fuente de barrido).}
\acro{TD--OCT}{\textit{Time-domain optical coherence tomography} (Tomografía de coherencia óptica en el dominio del tiempo).}
\acro{VISA}{\textit{Virtual instrument software architecture} (Arquitectura de software de instrumentos virtuales).}

\end{acronym}
\newpage
\thispagestyle{empty}



\makeatletter
\def\@afterheading{%
	\@nobreaktrue
	\everypar{%
		\if@nobreak
		\@nobreakfalse
		\clubpenalty \@M
		\hspace*{2em}%
		\else
		\clubpenalty \@clubpenalty
		\everypar{}%
		\fi}}
\makeatother

%%%%%%%%%%%%%%%%%%%%%%%%%%%%   Resumen
\newcommand \resumen{
\chapter*{\centering Resumen}

La \textit{espectroscopía} es el estudio de la materia a partir de su interacción con la radiación electromagnética. Su primera referencia en la ciencia moderna apareció en el trabajo \textit{Opticks} de Isaac Newton, desde entonces, la espectroscopía ha sido muy importante para el desarrollo de la física y una herramienta de gran útilidad en multiples aplicaciones relacionadas con la quimica, la biología y particularmente la óptica, como por ejemplo la identificación de elementos constituyentes de muestras desconocidas, el análisis de muestras biológicas y el estudio de la composición y dinámica de cuerpos celestes. El \textit{espectrómetro} es la herramienta básica de la espectroscopía, ya que permite medir el espectro de la luz. Una de sus aplicaciones se encuentra en tomografía óptica de coherencia (OCT), donde se emplea un espectrómetro como sistema de detección, una técncica de imagen médica \textit{no-invasiva} basada en interferometría de baja coherencia para producir imágenes de alta resolución del interior de la muestra. Este trabajo de grado consiste en el diseño e implementación a nivel de laboratorio de un espectrómetro lineal en el número de onda que opera en el espectro infrarrojo cercano para ser usado en la técnica OCT. Para lograr esto, se abordaron varios retos en áreas como el diseño óptico, la instrumentación óptica, la integración ópto-mecánica, la simulación y la programación. \\

En la primera parte de este trabajo se presenta un marco teórico que abarca conceptos y modelos físico--matamáticos relacionados con el funcionamiento de OCT y de un espectrómetro, indicando consideraciones de diseño de espectrómetros generales y relacionadas a OCT. Con base en esto, se realizó el diseño óptico del espéctrometro, con el uso del software OpticStudio (ZEMAX LLC), el cual incluye la definición del rango de funcionamiento y de la configuración óptica, la selección de los componentes ópticos, el análisis del desempeño óptico del sistema y la evaluación de la linealidad en el número de onda, la cual se obtiene con la combinación de una red de difracción y un prisma cuyas caracteristicas fueron escogidas mediante un procedimiento de optimización implementado en MATLAB (The MathWorks Inc.). Por otro lado, se propone un diseño mecánico preliminar para integrar los componentes ópticos, basado en un análisis de tolerancias mecánicas que se llevó a cabo en OpticStudio. Además, se realizó una evaluación del efecto de las cargas mecánicas que se aplican a los componentes principales mediante el método de elementos finitos, utilizando el software ANSYS Mechanical APDL (ANSYS Inc.). \\

En una segunda etapa, se realizó una prueba de concepto experimental que consistió en la alineación y calibración del espectrómetro en un banco óptico de laboratorio, con el fin de validar la linealidad en el número de onda. El protocolo de alineación establecido se basa en conceptos de óptica geométrica para alinear cada componente, ideas que pueden ser usadas en la alineación de otros espectrómetros o sistemas ópticos luego de una adecuación. El procedimiento de calibración se compone de dos partes, en una se emplean medidas experimentales de las líenas espcetrales de emisión del Argón y en la otra se utilizan curvas de respuesta espectral teoricas de los componentes. \\

El espectrómetro implementado opera en el rango espectral 6.321--8.585~$\mu$m$^{-1}$, equivalente a 994--732~$\mu$m, con una resolución de teórica de 1.106~mm$^{-1}$ (1.28~$\AA$) y una linealidad en el número de onda con un coeficiente de determinación de $R^2=0.999954$, lo que lo hace apto para ser usado en la técnica de tomografía óptica de coherencia. \\

\iffalse
Para lograr esto, se abordaron varios retos en áreas como el diseño óptico, la instrumentación óptica, la integración ópto-mecánica, la simulación y la programación.  se realizó el diseño óptico que incluyó la definición de la configuración óptica, la evaluación del desempeño del sistema y el análisis de tolerancias mecánicas que sirvieron como base de la definición de un diseño mecánico preliminar que se propone. Para la selección de los parámetros de los componentes ópticos se propone un procedimiento basado en la optimización de la linealidad en el número de onda utilizando el software MATLAB (The MathWorks Inc.). En una segunda etapa se realizó el montaje experimental del espectrómetro en un banco óptico, con el fin de validar la linealidad en el número de onda y establecer un protocolo de alineación, proceso en el cual se hizo uso de conceptos de óptica geométrica, apoyados en OpticStudio. Paralelamente se desarrolló un software de adquisición de datos utilizando el software LabVIEW (National Instruments Corp.), con el cual se digitaliza y almacena el espectro capturado. Este fue hecho teniendo en cuenta el funcionamiento futuro del sistema de SD-OCT. Además de la implementación a nivel de laboratorio. se realizó un diseño mecánico, en conjunto con el estudiante Santiago Garcia Botero, que sirve como propuesta para integrar el espectrometro en un sistema compacto en un trabajo futuro. El software ANSYS Mechanical APDL (ANSYS Inc.) se usó para analizar los efectos de las carga mecánicas en algunos componentes ópticos, producidas por las monturas mecánicas o las piezas diseñadas para acoplarlos.  \\

\textbf{Palabras claves:}.
\fi

}

\newpage
\resumen
%\addcontentsline{toc}{chapter}{\vspace*{-.5\baselineskip}\bf Resumen\vspace*{.5\baselineskip}}
\addcontentsline{toc}{chapter}{\bf Resumen}
\newpage



%%%%%%%%%%%%%%%%%%%%%%%%%%%%   Cuerpo del documento
\pagenumbering{arabic} % Ahora numerar páginascon números arábigos
\setcounter{page}{3} % Empezar el cotador de páginas en 3
\setcounter{secnumdepth}{4} % Empezar el cotador de páginas en 3

%%%%%%%%%%%%%%%%%%%%%%%%%%%%   Introducción
% \newpage
\phantomsection
\section{INTRODUCCIÓN}
a

%%%%%%%%%%%%%%%%%%%%%%%%%%%%   Referencias
\addcontentsline{toc}{chapter}{\bf Referencias}
\bibliographystyle{unsrt}
\bibliography{Bib_Source}

\end{document}