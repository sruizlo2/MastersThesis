\newpage
\phantomsection
\chapter{Introduction}

Light-matter interaction have become an important tool in medical sciences and biology, leading to the creation and development of a specialized field known as \textit{biomedical optics} or \textit{biophotonics} oriented to fundamental research, imaging, diagnosis, therapy and monitory of diseases and surgery assistance~\cite{}. Many imaging techniques have emerged to cover the general necessity to visualize internal structures of tissues, in particular, \textit{Optical Coherence Tomography} (OCT) has become an important imaging modality for biomedical optics and medicine. OCT \textit{is a non-invasive imaging technique that produces three dimensional micrometric-resolution images of scattering samples such as biological tissues by measuring the light that is backscattered by the sample using low-coherence interferometry}~\cite{}. OCT have gathered great interest in the research community in biophotonics given its unique features such its sensitive that allows to obtain useful information from biological samples with different optical properties, and its resolution of 1-15~$\mu$m and axial range of $\sim$2~mm that fills a gap between other medical imaging modalities such as ultrasound and confocal microscopy. Furthermore, non-invasive operation of OCT, both \exvi and \invi, with not contrast agents nor ionizing radiation are important features that have positioned OCT in the medical community for imaging of tissue pathologies \insi and in real time, particularly in ophthalmology~\cite{}, but also in intravascular imaging~\cite{}, endoscopic imaging~\cite{} and dermatology~\cite{}. 

\section{Optical coherence tomography}
OCT produces cross-sectional and volumetric images by measuring the magnitude and ``echo time delay'' of backscattered light from the sample, similarly to the operation of other tomographic techniques like ultrasound that employs sound instead of light. The backscattered light contains information of the optical properties of the sample and the information at different depths can be distinguished by determining the time it takes for light to travel different axial distances, thus performing an axial scan. Given the relatively large magnitude of the speed of light of $\sim$3x10$^8$m/s, there are technical limitations to make electronic devices with the required time resolution to measure the echo time delay of light with micro-metric resolution and high sensitivity. Because of that, OCT employs low-coherence interferometry to measure the backscattered light in terms of optical path length differences rather than temporal delays, being both related through the speed of light.

Providing cross-sectional images \insi and in real time without the need to remove and process specimens is an important feature of OCT for the visualization of tissue micro-structure and pathology. This possibility to perform ``optical biopsies'' enables operation of OCT in application where histopathology of excised tissue, the gold standard for assessing pathology, is insufficient for various reasons~\cite{}; (1) biopsy is hazardous or impossible, for example in the eye, arteries or nervous tissues, (2) biopsy is susceptible to sampling errors, given the impossibility to precisely detect the location of the pathology, for example in cancer diagnosis, leading to false negative, (3) real time visualization is required, for instance in guidance of invasive procedures, and (4) morphological information is not sufficient and additional functional measurements and imaging like blood flowmetry is necessary.


\section{Digital aberration correction in OCT}

\section{Problem statement}

\section{Objectives}

\subsection{General objectives}

\subsection{Specific objectives}