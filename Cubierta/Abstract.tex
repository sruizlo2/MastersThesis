\phantomsection
\addcontentsline{toc}{chapter}{\bf ABSTRACT}
\chapter*{ABSTRACT}
\markboth{}{ABSTRACT}

Optical coherence tomography (OCT) has become an important imaging modality for biomedical research and medical applications. OCT produces three-dimensional and high-resolution images of tissue by radiating light into the sample and measuring the backscattered light by means of interferometry. OCT signal is directly related to the complex-amplitude field of backscattered light and post-processing techniques has been very important to exploit the vast information contained in the signal, in order to improve visualization and analysis of images in terms of practical features like resolution, sensitivity and contrast, and also to provide additional types of optical contrasts that give extra and complementary information of the sample. Since it is an optical modality, OCT is prone to optical aberrations that degrade image quality and resolution, therefore correction of aberrations has been of great interest in OCT.

In the first part of this work, the ``OCT experiment'' is analyzed from an interference perspective to understand the basic operation of OCT that relies on optical interferometry. Then, an image formation model is described to explain the origin of aberrations in OCT imaging and how they can be corrected with computational state-of-the-art techniques. These approaches, however, are restricted to operate in specific system setups that provide the sufficient experimental conditions for their correct operation. The core of this work is to present a strategy that relaxes experimental conditions required to perform computational aberration correction in OCT tomograms. Additional post-processing tools are also proposed, in particular to reduce complex noise in the OCT signal, using non-local means to efficiently reduce noise by exploiting the tomographic information.

Computational aberration correction is demonstrated in experimental applications using different samples imaged with phase-unstable swept-source OCT systems that have been deemed unsuitable for this purpose in the past, given its experimental constrains to obtain reliable measurement of the complex signal. Correction of defocus is demonstrated in all experimental validations, showing resolution improvement beyond several times the Rayleigh range of the systems, up to 5 times in a proof of concept experiment. Significant improvement of image quality is presented in ocular, skin and endoscopic imaging, combining computational aberration with the additional tools developed here for noise reduction and post-processing techniques developed in the past for speckle noise suppression. Finally, computational refocusing of polarimetric properties of tissue measured with OCT is presented.